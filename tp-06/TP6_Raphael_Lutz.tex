\documentclass[10pt]{extarticle}

%\usepackage{algpseudocode}
\usepackage{amsmath}
\usepackage{amssymb}
%\usepackage[francais]{babel}
\usepackage{charter}
\usepackage{color}
\usepackage{comment}
\usepackage{empheq}
\usepackage{fancyhdr}
\usepackage[T1]{fontenc}
\usepackage[top=1.5cm, bottom=1.5cm, left=1.5cm, right=1.5cm]{geometry}
\usepackage{graphicx}
\usepackage[utf8]{inputenc}
%\usepackage{lipsum,booktabs}
%\usepackage{sectsty}
\usepackage{setspace}
\usepackage{titlesec}
%\usepackage{wrapfig}
%\usepackage{xcolor}
%\usepackage{listliketab}
%\usepackage{tabto}
\usepackage{amsmath}
\usepackage{pdflscape}
\usepackage{prftree}
\usepackage{pdfpages}

\definecolor{vertSec}{rgb}{0,0.3,0}
\definecolor{vertSSec}{rgb}{0,0.2,0}
\definecolor{vertSSSec}{rgb}{0,0,0}

\titleformat{\section}
{\color{vertSec}\normalfont\Large\bfseries}
{\color{vertSec}\thesection}{1em}{}

%\titleformat{name=\subsection}[block]
%{\color{vertSSec}\normalfont\Large\bfseries}
%{\hspace*{0.3cm}\thesubsection}
%{1em}
%{}

%\titleformat{name=\subsection,numberless}[block]
%{\color{vertSSec}\normalfont\normalsize\bfseries}
%{}
%{1pt}
%{\hspace*{0.3cm}}

\titleformat{\subsection}
{\color{vertSSec}\normalfont\large\bfseries}
{\color{vertSSec}\thesubsection}{1em}{}

\titleformat{\subsubsection}
{\color{vertSSSec}\normalfont\large\bfseries}
{\color{vertSSSec}\thesubsubsection}{1em}{}

%\newcommand{\itab}[1]{\hspace{0em}\rlap{#1}}
%\newcommand{\tab}[1]{\hspace{.2\textwidth}\rlap{#1}}

\begin{document}

%\pagestyle{fancy}
%\renewcommand{\headheight}{16pt}
%\lhead{Lutz Raphaël}
%\rhead{\today}
%\onehalfspacing

\section*{TP6 - Raphaël Lutz}

\subsection*{Constantes}
\begin{itemize}
\setlength\itemsep{0cm}
\item A1 (Alexandre)
\item A2 (Alexandrine)
\item R (Robin)
\item F (Floriane)
\item M (Miguel)
\end{itemize}

\subsection*{Variables}
\begin{itemize}
\setlength\itemsep{0cm}
\item $x$ ou $x_i$ une personne (implicitement inconnue)
\item $h$ ou $h_i$ une personne (implicitement un homme)
\item $f$ ou $f_i$ une personne (implicitement une femme)
\item $p$ ou $p_i$ une personne (implicitement un partenaire)
\end{itemize}

\subsection*{Prédicats}
\begin{itemize}
\setlength\itemsep{0cm}
\item $A(x,y)$ - ``x aime y'' (amour unilatéral)
\item $D(x,y)$ - ``x et y sont deux personnes distinctes''
\item $H(x,y)$ - ``x est un homme''
\item $F(x,y)$ - ``x est une femme''
\item $B(x,y)$ - ``x et y sont frères''
\end{itemize}

\subsection*{Formules logiques (saison 1)}

Par les points 2 et 3 de l'énoncé, nous savons que les personnes en couple s'aiment mutuellement. Ainsi, nous nous limiterons à un seul type de relation : le fait d'aimer quelqu'un ($A(x,y)$). Lors d'une relation de couple, nous écrirons $A(x,y) \land A(y,x)$. Nous allons aussi numéroter les variables le plus rigoureusement possible, afin d'éviter de futurs conflits.

\begin{flalign}
& A(A1,A2) \land A(A2,A1) \land A(R,F) \land A(F,R) & \\
& \exists h_1 \exists p_1 \exists x_1 \;\; H(h_1) \land A(h_1,p_1) \land A(p_1,h_1) \land D(p_1,x_1) \land A(h_1,x_1) & \\
& \exists f_2 \exists p_2 \exists x_2 \;\; F(f_2) \land A(f_2,p_2) \land A(p_2,f_2) \land D(p_2,x_2) \land A(f_2,x_2) & \\
& \exists h_3 \exists p_3 \forall x_3 \;\; H(h_3) \land A(h_3,p_3) \land A(p_3,h_3) \land D(p_3,x_3) \land \neg A(h_3,x_3) & \\
& \exists f_4 \exists p_4 \forall x_4 \;\; F(f_4) \land A(f_4,p_4) \land A(p_4,f_4) \land D(p_4,x_4) \land \neg A(f_4,x_4) & \\
& \exists x_{5} \exists x_{5'} \;\; A(M,x_5) \land A(x_5,x_{5'}) \land A(x_{5'},A2) & \\
& \forall f_6 \forall f_{6'} \;\; F(f_6) \land F(f_{6'}) \land \neg A(f_6,f_{6'}) & \\
& \exists x_7 \exists x_{7'} \exists x_{7''} \;\; A(R,x_7) \land A(x_7,x_{7'}) \land A(x_{7'},x_{7''}) \land A(x_{7''},x_7) & \\
& \forall x_8 \;\; \neg A(x_8,x_8) &
\end{flalign}

\emph{Le graphe sera joint au pdf en dernière page. J'ai précisé les relations connues des relations supposées.}

\subsection*{Formules logiques (saison 2)}

\begin{flalign}
& \exists b_9 \;\; B(b_9,A1) & \\
& \exists i_{10} \exists i_{10'} \;\; B(i_{10},i_{10'}) \land (A(i_{10},i_{10'}) \lor A(i_{10'},i_{10})) & \\
& \neg A(M,F) &
\end{flalign}


\newpage
\begin{landscape}
\subsection*{Calcul des séquents}
On utilise la convention suivante : on regroupe les 10 premières formules sous la dénomination $\Gamma$. Lorsque nous avons besoin d'extraire l'une des formules (1 à 10) de $\Gamma$, on utilise la règle de contraction gauche (noté CG, suivi du numéro de la formule extraite, e.g. CG2) :

\begin{displaymath}
\prftree[r]{(CG)}{\Gamma , A , A \vdash \Delta}{\Gamma, A \vdash \Delta}
\end{displaymath}

\begin{center}
\emph{[source : Wikipédia, Calcul des séquents]}
\end{center}

Cette règle (triviale mais très utile dans notre cas), nous garantis que $\Gamma$ reste identique, et nous permet d'extraire les formules souhaitées sans devoir ré-exprimer $\Gamma$ ou le modifier.

La conclusion du jugement seront les formules (11) et (12), où nous souhaitons démontrer que l'une des deux doit être juste (i.e. inceste ou Miguel n'aime pas Floriane). Nous pouvons donc commencer le calcul des séquents.

\begin{displaymath}
\prftree[r]
{($\exists$r,$\exists$r)}
{\prftree[r]{($\neg$r)}
{\prftree[r]{($\land$r)}
{\prftree[r]{(CG10)}
{\prftree[r]{($\exists$l)}
{\prftree[r]{basic}
{\text{\textcircled{1}}}
{\prftree[r]{(déduction)}
{\Gamma , A(M,F) , B(b_9,A1) \vdash B(t_{10},A1)}
{\Gamma , A(M,F) , B(b_9,A1) \vdash B(t_{10},t_{10'})}}}
{\Gamma , A(M,F) , \exists b_9 \; B(b_9,A1) \vdash B(t_{10},t_{10'})}}
{\Gamma , A(M,F) \vdash B(t_{10},t_{10'})}}
{\prftree[r]{}
{\text{\textcircled{2}}}
{\Gamma , A(M,F) \vdash A(t_{10},t_{10'}) \lor A(t_{10'},t_{10})}}
{\Gamma , A(M,F) \vdash B(t_{10},t_{10'}) \land (A(t_{10},t_{10'}) \lor A(t_{10'},t_{10}))}}
{\Gamma \vdash B(t_{10},t_{10'}) \land (A(t_{10},t_{10'}) \lor A(t_{10'},t_{10})), \neg A(M,F)}}
{\Gamma \vdash \exists i_{10} \exists i_{10'} \; B(i_{10},i_{10'}) \land (A(i_{10},i_{10'}) \lor A(i_{10'},i_{10})), \neg A(M,F)}
\end{displaymath}

On observe que $t_{10'} = A1$, et ainsi la branche \textcircled{1} devient un basic. Afin de simplifier les calculs, continuons maintenant en nous intéressant uniquement à la branche \textcircled{2}, en sachant maintenant que la personne $t_{10'}$ est Alexandre :

\begin{displaymath}
\prftree[r]{(rappel)}
{\prftree[r]{($\scriptstyle t_{10'}=A1$)}
{\prftree[r]{(CG2)}
{\prftree[r]{($\exists$l,$\exists$l,$\exists$l)}
{\prftree[r]{($\land$l,$\land$l,$\land$l,$\land$l,$\lor$r)}
{\prftree[r]{($\lor l$)}
{\prftree[r]{}
{\prftree[r]{}
{\text{\textcircled{3}}}
{\Gamma , A(M,F) , D(A2,x_1) , A(A1,x_1) \vdash A(t_{10},A1) , A(A1,t_{10})}}
{\Gamma , A(M,F) , A(A1,A2) , A(A2,A1) , D(A2,x_1) , A(A1,x_1) \vdash A(t_{10},A1) , A(A1,t_{10})}}
{\prftree[r]{}
{\prftree[r]{}
{\text{\textcircled{4}}}
{\Gamma , A(M,F) , D(F,x_1) , A(R,x_1) \vdash A(t_{10},A1) , A(A1,t_{10})}}
{\Gamma , A(M,F) , A(R,F) , A(F,R) , D(F,x_1) , A(R,x_1) \vdash A(t_{10},A1) , A(A1,t_{10})}}
{\Gamma , A(M,F) , H(h_1) , A(h_1,p_1) , A(p_1,h_1) , D(p_1,x_1) , A(h_1,x_1) \vdash A(t_{10},A1) , A(A1,t_{10})}}
{\Gamma , A(M,F) , H(h_1) \land A(h_1,p_1) \land A(p_1,h_1) \land D(p_1,x_1) \land A(h_1,x_1) \vdash A(t_{10},A1) \lor A(A1,t_{10})}}
{\Gamma , A(M,F) , \exists h_1 \exists p_1 \exists x_1 \; H(h_1) \land A(h_1,p_1) \land A(p_1,h_1) \land D(p_1,x_1) \land A(h_1,x_1) \vdash A(t_{10},A1) \lor A(A1,t_{10})}}
{\Gamma , A(M,F) \vdash A(t_{10},A1) \lor A(A1,t_{10})}}
{\Gamma , A(M,F) \vdash A(t_{10},t_{10'}) \lor A(t_{10'},t_{10})}}
{\text{\textcircled{2}}}
\end{displaymath}

Au dernier étage du calcul de séquent, nous remarquons que l'homme infidèle (formule (2)) ne peut qu'être Alexandre ou Robin. Nous pouvons donc diviser l'arbre en 2 suivant ces deux personnes, et cela nous donne directement une indication de la partenaire (respectivement Alexandrine et Floriane). De plus, comme nous retrouvons les éléments de la formule (1) i.e. $A(A1,A2) \land A(A2,A1)$ et $A(R,F) \land A(F,R)$, nous pouvons également les enlever (puisqu'elles sont toujours contenues dans le $\Gamma$ !). Introduisons une nouvelle règle, nommée doublon (triviale mais utile pour la lisibilité) :

\begin{displaymath}
\prftree[r]{(doublon)}
{A \vdash \Delta}
{A,A \vdash \Delta}
\end{displaymath}

Ceci nous a donc amené à ces deux nouvelles branches \textcircled{3} et \textcircled{4}. Commençons par la branche \textcircled{3} en utilisant la formule (4) qui est complémentaire à la formule (2), puisque dans ce cas, l'homme fidèle est donc Robin, tandis que dans la branche \textcircled{4} l'homme fidèle est Alexandre. Nous nous permettons maintenant d'enlever directement les $\exists$ gauches (puisque les noms de variables ont été choisis avec soin), ainsi que de remplacer les $\land$ gauches par des virgules (règle $\land$r) si besoin. Il est également important de se souvenir qu'à chaque supposition, nous créons plusieurs branches dans le calcul des séquents. Ainsi, à chaque règle de la forme ($x_i=$personne), il faut se souvenir que \textbf{nous n'explorons qu'une de toutes les branches possibles}. Par soucis de clarté, je me permettrai donc de ne pas exprimer les autres branches explicitement.

\subsubsection*{Branche \textcircled{3}}

\begin{displaymath}
\prftree[r]{(rappel)}
{\prftree[r]{(CG4)}
{\prftree[r]{($\forall$l)}
{\prftree[r]{($\land$l,doublon)}
{\prftree[r]{}
{\text{abandon (et erreur fort probable ...)}}
{\Gamma , D(F,t_3) , \neg A(R,t_3) , A(M,F) , D(A2,x_1) , A(A1,x_1) \vdash A(t_{10},A1) , A(A1,t_{10})}}
{\Gamma , A(R,F) \land A(F,R) \land D(F,t_3) \land \neg A(R,t_3) , A(M,F) , D(A2,x_1) , A(A1,x_1) \vdash A(t_{10},A1) , A(A1,t_{10})}}
{\Gamma , \forall x_3 A(R,F) \land A(F,R) \land D(F,x_3) \land \neg A(R,x_3) , A(M,F) , D(A2,x_1) , A(A1,x_1) \vdash A(t_{10},A1) , A(A1,t_{10})}}
{\Gamma , A(M,F) , D(A2,x_1) , A(A1,x_1) \vdash A(t_{10},A1) , A(A1,t_{10})}}
{\text{\textcircled{3}}}
\end{displaymath}

\subsubsection*{Branche \textcircled{4}}

\begin{displaymath}
\prftree[r]{(rappel)}
{\prftree[r]{(CG4,$\exists$l,$\exists$l)}
{\prftree[r]{($\forall$l,$\land$l,$\land$l,$\land$l,doublon)}
{\prftree[r]{(CG3,$\exists$l,$\exists$l)}
{\prftree[r]{($\land$l,$\land$l,$\land$l)}
{\prftree[r]{($\scriptstyle f_2=F,p_2=R$,doublon)}
{\prftree[r]{($\scriptstyle x_1=M,x_2=A1$)}
{\prftree[r]{(CG5,$\exists$l,$\land$l)}
{\prftree[r]{(CG6,$\exists$l,$\land$l)}
{\prftree[r]{($\scriptstyle x_5=F,x_{5'}=A1$,doublon)}
{\prftree[r]{(CG8,$\exists$l,$\land$l)}
{\prftree[r]{($\scriptstyle x_7=M$,doublon)}
{\prftree[r]{($\scriptstyle x_{7'}=F$,doublon)}
{\prftree[r]{($\scriptstyle x_{7''}=R$,doublon)}
{\prftree[r]{$\neg$basic}
{}
{\Gamma , D(A_1,t_4) , \neg A(A2,t_4) , A(F,A1) , D(A2,t_3) , \neg A(A1,t_3) , A(M,F) , A(R,M) \vdash A(t_{10},A1) , A(A1,t_{10})}}
{\Gamma , A(F,x_{7''}) , A(x_{7''},M) , D(A_1,t_4) , \neg A(A2,t_4) , A(F,A1) , D(A2,t_3) , \neg A(A1,t_3) , A(M,F) , A(R,M) \vdash A(t_{10},A1) , A(A1,t_{10})}}
{\Gamma , A(M,x_{7'}) , A(x_{7'},x_{7''}) , A(x_{7''},M) , D(A_1,t_4) , \neg A(A2,t_4) , A(F,A1) , D(A2,t_3) , \neg A(A1,t_3) , A(M,F) , A(R,M) \vdash A(t_{10},A1) , A(A1,t_{10})}}
{\Gamma , A(R,x_7) , A(x_7,x_{7'}) , A(x_{7'},x_{7''}) , A(x_{7''},x_7) , D(A_1,t_4) , \neg A(A2,t_4) , A(F,A1) , D(A2,t_3) , \neg A(A1,t_3) , A(M,F) , A(R,M) \vdash A(t_{10},A1) , A(A1,t_{10})}}
{\Gamma , D(A_1,t_4) , \neg A(A2,t_4) , A(F,A1) , D(A2,t_3) , \neg A(A1,t_3) , A(M,F) , A(R,M) \vdash A(t_{10},A1) , A(A1,t_{10})}}
{\Gamma , A(M,x_5) , A(x_5,x_{5'}) , A(x_{5'}, A2) , D(A_1,t_4) , \neg A(A2,t_4) , A(F,A1) , D(A2,t_3) , \neg A(A1,t_3) , A(M,F) , A(R,M) \vdash A(t_{10},A1) , A(A1,t_{10})}}
{\Gamma , D(A_1,t_4) , \neg A(A2,t_4) , A(F,A1) , D(A2,t_3) , \neg A(A1,t_3) , A(M,F) , A(R,M) \vdash A(t_{10},A1) , A(A1,t_{10})}}
{\Gamma , A(F,A1) , D(A2,t_3) , \neg A(A1,t_3) , A(M,F) , A(R,M) \vdash A(t_{10},A1) , A(A1,t_{10})}}
{\Gamma , D(R,x_2) , A(F,x_2) , D(A2,t_3) , \neg A(A1,t_3) , A(M,F) , D(F,x_1) , A(R,x_1) \vdash A(t_{10},A1) , A(A1,t_{10})}}
{\Gamma , F(f_2) , A(f_2,p_2) , A(p_2,f_2) , D(p_2,x_2) , A(f_2,x_2) , D(A2,t_3) , \neg A(A1,t_3) , A(M,F) , D(F,x_1) , A(R,x_1) \vdash A(t_{10},A1) , A(A1,t_{10})}}
{\Gamma , F(f_2) \land A(f_2,p_2) \land A(p_2,f_2) \land D(p_2,x_2) \land A(f_2,x_2) , D(A2,t_3) , \neg A(A1,t_3) , A(M,F) , D(F,x_1) , A(R,x_1) \vdash A(t_{10},A1) , A(A1,t_{10})}}
{\Gamma , D(A2,t_3) , \neg A(A1,t_3) , A(M,F) , D(F,x_1) , A(R,x_1) \vdash A(t_{10},A1) , A(A1,t_{10})}}
{\Gamma , \forall x_3 A(A1,A2) \land A(A2,A1) \land D(A2,x_3) \land \neg A(A1,x_3) , A(M,F) , D(F,x_1) , A(R,x_1) \vdash A(t_{10},A1) , A(A1,t_{10})}}
{\Gamma , A(M,F) , D(F,x_1) , A(R,x_1) \vdash A(t_{10},A1) , A(A1,t_{10})}}
{\text{\textcircled{4}}}
\end{displaymath}

\end{landscape}
\newpage

Nous arrivons enfin à la partie intéressante. Nous voyons que nous avons développé toutes les formules dignes d'intérêt : (2),(3),(4),(5),(6),(8),(10),(11),(12). La formule (1) n'a pas d'utilité propre hormis pour faire des suppositions qui ont un intérêt, tout comme les formules (7) et (9). Or, la partie droite nous dit qu'une des deux possibilités suivantes doit être remplie :

\begin{itemize}
\item ``Un homme (frère caché d'Alexandre) aime Alexandre'' $A(t_{10},A1)$
\item ``Alexandre aime un homme (qui est son frère caché'' $A(A1,t_{10})$
\end{itemize}

Il est impossible qu'Alexandre aime une autre personne, d'après $D(A2,t_3) \land \neg A(A1,t_3)$. Et aucun homme n'aime Alexandre, puisque $A(M,F)$ et $A(R,M)$ sont les deux seules relations qu'on des hommes envers d'autres hommes. Cela nous mène donc à un non-basic, et donc nous ne pouvons pas prouver à notre ``ami'' (de l'énoncé), que soit il existe une relation incestueuse soit que Miguel n'aime pas Floriane.

\vspace{0.6cm}

J'aurais également aimé trouver un moyen plus léger de faire cette preuve, et je m'excuse d'avance pour la lourdeur de ces lignes de calculs ... Voici finalement un résumé de la structure totale du calcul de séquent, pour aider à la compréhension générale de la structure que j'ai utilisé

\begin{displaymath}
\prftree[r]{}
{\prftree[r]{}
{\prftree[r]{}
{\;\;\;\;\;\text{\textcircled{1}}\;\;\;\;\;}
{...}}
{\prftree[r]{}
{\prftree[r]{}
{\prftree[r]{}
{\prftree[r]{}
{\;\;\;\;\;\text{\textcircled{3}}\;\;\;\;\;}
{...}}
{\prftree[r]{}
{\;\;\;\;\;\text{\textcircled{4}}\;\;\;\;\;}
{...}}
{...}}
{\;\;\;\;\;\text{\textcircled{2}}\;\;\;\;\;}}
{...}}
{...}}
{\;\;\;\;\;\Gamma \vdash \Delta\;\;\;\;\;}
\end{displaymath}


\includepdf[pages={1}]{TP6_graphe.pdf}

\end{document}
